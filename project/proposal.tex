% Term Project: Proposal
% CS457 Functional Programming
% Professor: Jones
% Russell Miller
% Feb 15 2012

\documentclass{article}
\usepackage{anysize}
\usepackage{wasysym}
\usepackage{graphicx}

\marginsize{2cm}{2cm}{2cm}{2cm}

\title{Term Project: Proposal\\
CS457 Functional Programming\\
with Professor Mark Jones\\
Winter 2012}
\author{Russell Miller}
\date{\today}

\begin{document}

\maketitle

\section*{Abstract}
My goal is to work with a Haskell library to create MIDI drum tracks. In doing 
my own research, as well as speaking with you (Mark), I've discovered the 
Haskore Computer Music System.\\
\\
I'd like to dive in and attempt to create some interesting music samples. I 
imagine this will not take very much work. I've already installed Haskore on my
development laptop. Once I've determined the basics, I'd 
like to simplify the steps by writing a Haskell program that works as a wrapper
to abstract the design of the tracks. I'd like to focus on things that 
differentiate styles of drum parts.\\
\\
Being a guitar player and not a drummer I'm not sure how good the quality will 
be, musically speaking.

\section*{Motivation (A Side Note)}
I feel like there's value in reiterating why I chose this topic. It was somewhat
of an epiphany. I have a project for a different class where I'm attempting to
do a simple music production by recording and editing some audio. One of the 
ideas I had along the way was to record myself on guitar, and then use a MIDI
keyboard to add a drum part.\\
\\
However, since Haskore seems to be capable I would rather do something more
programming-related and come up with the drum track using a Haskell library. I
am really hoping my other teacher (not a CS course) is impressed.

\section*{Some Initial Thoughts}
The first thing I did is set this up on my development machine, which is a Macbook
running OSX Lion (10.7.2). The steps to do so (using cabal) are outlined on the Haskore 
website\footnote{http://www.haskell.org/haskellwiki/Haskore}.\\
\\
The installation steps were uneventful. Moving on to the simple example, I put that into
a Haskell source file and tried to load it. They don't tell you exactly what part of the
Haskore system to import so I got stuck at that point.\\
\\
On the Haskore website there are also helpful links, and I followed the link to
``A Dummy's Guide to Haskore''\footnote{https://github.com/nfjinjing/haskore-guide/blob/master/doc/index.markdown}.
This guide is written in Markdown, and Github is able to parse that just fine. The first part
of this guide helps you make sure Haskore is set up correctly. It turns out I wasn't quite
ready, because GHCi couldn't find the Haskore module. Some searching resulted in discovering
I also needed to install the ``haskore-vintage'' package.
\begin{verbatim}
$ cabal install haskore-vintage
\end{verbatim}
I proceeded to the ``Hello World'' example and tried it out and it worked. 

\section*{Scaling Potential}
We talked in class about being able to scale. The first thing that comes to my mind
for a way to scale this \emph{down} is to turn this into more of a Haskore tutorial,
demoing what I was able to do. If I am working on creating a good track and I run out
of time I can simply outline the steps I took.\\
\\
A way to scale this project up could be to add some sort of fancy front end to it, possibly. 
If I get done in an hour or two and my work feels trivial I'll talk to you more about how to
beef it up.

\section*{Timeline}
Over the next 5 weeks I will need to accomplish some significant tasks. My end goal is to
have some sort of Haskell program that can create MIDI drum tracks. Right now I was just able
to create a MIDI file that plays a single note.
\subsection*{Week 1 - Simple Drum Instrumental}
The single note that the ``Hello World'' example plays is on a Grand Piano. I will need to
figure out how to do drum sounds and hear what the different notes sound like. I imagine
it will take some time to get through the list of 128 sounds that are provided by the MIDI
standard and pick the ones I want to use.
\subsection*{Week 2 - Tempo}
I'm not sure how tempo works with Haskore, but I need to spend some time figuring out how I
can use tempo as a parameter for my final program. If you can only create drum tracks at one
speed they won't be able to fit together with other instruments very well.
\subsection*{Week 3 - A Few Good Songs}
I'll try to get some input from a friend of mine that plays drums on what some standard
drumming patterns are, to attempt to create MIDI tracks like them. Once I've done that, I'll analyze
how I can parameterize my program to customize songs in this way.
\subsection*{Week 4 \& 5 - Final Touch Ups}
At this point I will have a better idea exactly how the final product is going to look.
It will take the rest of the time to get it polished and presentable.

\end{document}
